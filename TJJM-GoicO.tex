%=========================================================================%
%| %%—————————————————————————————————————————————————————————————————%% |%
%| %                 author:GoicO | SoiYcn                             % |%
%| %                 datetime:2024/3/20 15:00                          % |%
%| %                 email:17802018010@163.com                         % |%
%| %%—————————————————————————————————————————————————————————————————%% |%
%=========================================================================%





\documentclass[a4paper]{ctexart}
\usepackage{goico}


% ——————————————————————————————可选设定  ————————————————————————————————————
% \usepackage{cite}
% \usepackage[style=gb7714-2015ay, backend=biber]{biblatex}
% \usepackage[numbers]{natbib}% 可选,提供更多引用命令选项
% \bibliographystyle{gbt7714-numerical}% 设置参考文献样式
% ——————————————————————————————可选设定  ————————————————————————————————————


\cbooktitle{2024 年(第十届)全国大学生统计建模大赛}
\ctitle{人工智能时代大数据产能统计研究}
\cauthor{周捷、杨景成、蓝玉茜}
\cteacher{钟晓君}
\cschool{广东技术师范大学}



\begin{document}    %文档的开始,一定要有文档的结束,才能生效


%-------------------------封面部分 起-------------------------
\makecoverpage
%-------------------------封面部分 止-------------------------


%-------------------------摘要部分 起-------------------------
\begin{titlepage}
    \makeabstractstart
    \begin{abstract}\abstractcontent
    最小二乘回归模型通过最小化误差的平方和寻找数据的最佳函数匹配,可以简便地求得未知的数据,并使得这些求得的数据与实际数据之间误差的平方和为最小,因此本文选用最小二乘回归模型作为矫正模型。我们将$\frac{2}{3}$ 的数据用于训练集合,剩下$\frac{1}{2}$的数据用于检测集合。我们发现吸光度-波长图像下方的数据预测效果良好,上方的数据预测效果。\par
    最后,我们对本模型进行评价,提出了可以用偏最小二乘、机器学习算法进行预测的展望。\par
    {\keyword{关键词:}}红外光谱;Savitzky-Golay平滑处理;混合溶液浓度检测;预测模型;普通最小二乘回归模型
    \addcontentsline{toc}{section}{摘要} % 将“摘要”添加到目录(根据需要调整为 section 等)
    \end{abstract}
\end{titlepage}
%-------------------------摘要部分 止-------------------------



%-------------------------目录部分 起-------------------------
\makecontent
%-------------------------目录部分 止-------------------------



%=========================正文部分 起=========================

%-------------------------正文开始-------------------------
\maketxtstart
%-------------------------正文开始-------------------------


%-------------------------问题重述-------------------------

\section{问题重述}
目前有两种溶液U(VI)和U(VI)按照一定的浓度与硝酸溶液混合。经过红外光线扫描,得到该混合溶液在一定波长下的频谱。试根据现有频谱数据建立预测两种溶液的浓度的数学模型,并用该模型预测出 \cite{hariharan2014simultaneous} 待检混合溶液样本中的两种溶液的浓度。
目前有两种溶液U(VI)和U(VI)按照一定的浓度与硝酸溶液混合。经过红外光线扫描 \cite{周222222},得到该混合溶液在一定波长下的频谱。试根据现有频谱数据建立预测两种溶液的浓度的数学模型,并用该模型预测出待检混合溶液样本中的两种溶液的浓度。
目前有两种溶液U(VI)和U(VI)按照一定的浓度与硝酸溶液混合。经过红外光线扫描,得到该混合溶液在一定波长下的频谱。试根据现有频谱数据建立预测两种溶液的浓度的数学模型,并用该模型预测出待检混合溶液样本中的两种溶液的浓度 \cite{周2121222}。


\begin{figure}[ht]
  \centering
  % \fcolorbox{MSBlue}{white}
  % {\includegraphics[width=0.7\textwidth]{./Img/Guang}}
  \includegraphics[width=0.7\textwidth]{./Img/Guangdong101.jpg}
  \caption{Pixel 5 型号虚拟机}\label{fig:a_0}
\end{figure}


\subsection{巴萨神风}

\begin{figure}[ht]
  \centering
  % \fcolorbox{MSBlue}{white}
  % {\includegraphics[width=0.7\textwidth]{./Img/Guang}}
  \includegraphics[width=0.7\textwidth]{./Img/Guangdong101.jpg}
  \caption{Pixel 5 型号虚拟机}\label{fig:a_1}
\end{figure}

\subsection{欧阳}

\subsubsection{小巴萨}

% \begin{figure}[ht]
%   \centering
%   % \fcolorbox{MSBlue}{white}
%   % {\includegraphics[width=0.7\textwidth]{./Img/Guang}}
%   \includegraphics[width=0.7\textwidth]{./Img/Guangdong101.jpg}
%   \caption{Pixel 5 型号虚拟机}\label{fig:a_2}
% \end{figure}

\subsubsection{小巴萨}


使用了某种特殊文档类或宏包,它们改变了默认的页面布局;
页面设置(如  设置)不够合理,导致页眉区域与正文内容重叠;
如果您尝试自定义标题样式,并且没有正确处理垂直空间,也可能导致标题位置异常;
使用了一些控制标题定位的命令,比如  或相关宏包(如 titlesec)中的一些选项,调整不当也可能引起问题。
要解决这个问题,请检查文档中相关的设置部分,并确保没有影响页面布局的冲突命令。如果需要section标题紧随正文内容而非页眉,可以尝试不修改默认样式或者查阅所使用的文档类或宏包的文档来获得正确的定制方法。如需手动调整,可以考虑使用 titlesec 宏包等工具来精确控制各层级标题的位置和格式。如果问题依旧存在,请提供具体的代码片段以便进一步分析和解决问题。

\subsubsection{小巴萨}

您的问题涉及两个不同的方面:一个是关于参考文献样式的设置,另一个是关于section标题的布局问题。

对于参考文献样式,您已经正确地引入了 natbib 宏包以提供更多的引用命令选项,并设置了符合 GB/T 7714 标准的 gbt7714-numerical 样式,这样可以确保参考文献按照国标要求进行格式化。

至于section标题出现在页面的空白处且与页码重叠的问题,这通常是因为文档类或自定义设置不当造成的布局问题。在 LaTeX 中,section 标题的默认位置是在页面顶部开始的地方,除非被设计成出现在特定的位置(比如页眉)或者受到其它宏包的影响。

如果您的section标题出现在页码所在的位置,可能的原因包括:

\subsubsection{小巴萨}

您的问题涉及两个不同的方面:一个是关于参考文献样式的设置,另一个是关于section标题的布局问题。

对于参考文献样式,您已经正确地引入了 natbib 宏包以提供更多的引用命令选项,并设置了符合 GB/T 7714 标准的 gbt7714-numerical 样式,这样可以确保参考文献按照国标要求进行格式化。

至于section标题出现在页面的空白处且与页码重叠的问题,这通常是因为文档类或自定义设置不当造成的布局问题。在 LaTeX 中,section 标题的默认位置是在页面顶部开始的地方,除非被设计成出现在特定的位置(比如页眉)或者受到其它宏包的影响。

如果您的section标题出现在页码所在的位置,可能的原因包括:

\subsubsection{小巴萨}

您的问题涉及两个不同的方面:一个是关于参考文献样式的设置,另一个是关于section标题的布局问题。

对于参考文献样式,您已经正确地引入了 natbib 宏包以提供更多的引用命令选项,并设置了符合 GB/T 7714 标准的 gbt7714-numerical 样式,这样可以确保参考文献按照国标要求进行格式化。

至于section标题出现在页面的空白处且与页码重叠的问题,这通常是因为文档类或自定义设置不当造成的布局问题。在 LaTeX 中,section 标题的默认位置是在页面顶部开始的地方,除非被设计成出现在特定的位置(比如页眉)或者受到其它宏包的影响。

如果您的section标题出现在页码所在的位置,可能的原因包括:

\subsubsection{小巴萨}
\subsubsection{小巴萨}


\subsection{欧阳}

\subsubsection{小巴萨}
\subsubsection{小巴萨}
\subsubsection{小巴萨}
\subsubsection{小巴萨}

\subsection{欧阳}
\subsection{欧阳}
\subsection{欧阳}
\subsection{欧阳}


\section{问题重述}

\subsection{欧阳}

\subsubsection{小巴萨}
\subsubsection{小巴萨}
\subsubsection{小巴萨}
\subsubsection{小巴萨}
\subsubsection{小巴萨}




您的问题涉及两个不同的方面:一个是关于参考文献样式的设置,另一个是关于section标题的布局问题。

对于参考文献样式,您已经正确地引入了 natbib 宏包以提供更多的引用命令选项,并设置了符合 GB/T 7714 标准的 gbt7714-numerical 样式,这样可以确保参考文献按照国标要求进行格式化。

至于section标题出现在页面的空白处且与页码重叠的问题,这通常是因为文档类或自定义设置不当造成的布局问题。在 LaTeX 中,section 标题的默认位置是在页面顶部开始的地方,除非被设计成出现在特定的位置(比如页眉)或者受到其它宏包的影响。

如果您的section标题出现在页码所在的位置,可能的原因包括:

使用了某种特殊文档类或宏包,它们改变了默认的页面布局;
页面设置(如  设置)不够合理,导致页眉区域与正文内容重叠;
如果您尝试自定义标题样式,并且没有正确处理垂直空间,也可能导致标题位置异常;
使用了一些控制标题定位的命令,比如  或相关宏包(如 titlesec)中的一些选项,调整不当也可能引起问题。
要解决这个问题,请检查文档中相关的设置部分,并确保没有影响页面布局的冲突命令。如果需要section标题紧随正文内容而非页眉,可以尝试不修改默认样式或者查阅所使用的文档类或宏包的文档来获得正确的定制方法。如需手动调整,可以考虑使用 titlesec 宏包等工具来精确控制各层级标题的位置和格式。如果问题依旧存在,请提供具体的代码片段以便进一步分析和解决问题。

\section{问题重述}



\begin{table}[t]
    \centering
    \caption{字体使用规范}
    \begin{tabular}{|c|c|}
        \hline
        论文题目               & 黑体二号居中                                  \\ \hline
        中文摘要标题           & 黑体三号居中                                  \\ \hline
        中文摘要内容           & 宋体小四号                                    \\ \hline
        中文关键词             & 宋体小四号(标题``关键词''加粗)              \\ \hline
        英文摘要标题           & Times New Roman加粗三号全部大写               \\ \hline
        英文摘要内容           & Times New Roman小四号                         \\ \hline
        英文关键词             & Times New Roman小四号(标题``Keywords''加粗) \\ \hline
        目录标题               & 黑体三号居中                                  \\ \hline
        目录内容               & 宋体小四号                                    \\ \hline
        正文各章标题           & 黑体三号居中                                  \\ \hline
        正文各节一级标题       & 黑体四号左对齐                                \\ \hline
        正文各节二级及以下标题 & 宋体小四号加粗左对齐空两格                    \\ \hline
        正文内容               & 宋体小四号                                    \\ \hline
        参考文献标题           & 黑体三号居中                                  \\ \hline
        参考文献内容           & 宋体五号                                      \\ \hline
        致谢、附录标题         & 黑体三号居中                                  \\ \hline
        致谢、附录内容         & 宋体小四号                                    \\ \hline
        页眉与页脚             & 宋体五号居中                                  \\ \hline
        图题、表题             & 宋体五号                                      \\ \hline
        脚注、尾注             & 宋体小五号                                    \\ \hline
    \end{tabular}
    \label{tab:font-spec}
\end{table}






%=========================正文部分 止=========================




%-------------------------参考文献 起-------------------------
\makereferences
%-------------------------参考文献 止-------------------------





%-------------------------附录部分 起-------------------------
\begin{slsappendix}
    这里是附录部分,随便写点啥都可以的,啊吧啊吧啊吧八八八八八八八八八八八八

    
本模版支持在论文中插入代码片段,或直接从源码文件进行插入。
例如,在论文中插入代码片段的效果为:
\begin{python}
    def func():
    print("hello world")
    with open('./output.txt', 'w') as f:
    L = f.readlines()

    if __name__ == "__main__":
    # this is a comment line
    func()
\end{python}
也可在行内插入代码片段,例如:Python中重载加法运算符的函数为\pyinline{__add__},类的标识符为\pyinline{class}。
此外,还可直接插入代码文件,例如插入\texttt{./code/demo.cpp}的效果为:
\lstinputlisting[style=cppstyle]{code/demo.cpp}




\end{slsappendix}
% %-------------------------附录部分 止-------------------------



%-------------------------致谢部分 起-------------------------
\begin{acknowledgments}
    我谨向我的导师表示衷心感谢,他们的悉心指导和支持使我在科研道路上不断前行。
    
    同时,我也要感谢我的同学们,在项目合作期间给予的无私帮助与宝贵建议。

    最后,我要感谢家人在我学术生涯中的陪伴与鼓励。

\end{acknowledgments}
%-------------------------致谢部分 止-------------------------


\end{document}



